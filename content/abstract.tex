% !TEX root = ../thesis.tex

\makeatletter
\def\input@path{{../}}
\makeatother
\documentclass[../thesis.tex]{subfiles}

\begin{document}

\chapter*{ABSTRACT}%
%\parindent \z@ \centering 
\normalfont
%{\large\textbf{\abstractname}} \\
\vspace{1.0cm}
\begin{singlespacing}%
%\vskip 20\p@
\addcontentsline{toc}{chapter}{ABSTRACT}

Saat ini teknologi \textit{deep learning} telah banyak digunakan dalam berbagai bidang mulai dari pengolahan citra, pengenalan suara, klasifikasi bahkan deteksi suatu objek. Telah banyak jaringan deep learning yang dikembangkan. Untuk deteksi objek terdapat
YOLO (\textit{You Only Look Once}), SSD, dan Fast-RCNN sedangkan untuk data yang sifatnya sekuensial waktu terdapat LSTM, RNN dan GRU. 

Pada penelitian ini akan dibahas mengenai perancangan sebuah sistem agar dapat melakukan prediksi arus lalu lintas secara real-time menggunakan kamera CCTV jalan raya. Pendekatan yang dilakukan berbasis deep learning dengan menggunakan beberapa arsitektur yang telah ada.
Secara garis besar, proses perancangan sistem akan dibagi menjadi 3 bagian yaitu perancangan objek detektor menggunakan arsitektur YOLOV3 untuk diimplementasikan di CCTV, perancangan LSTM untuk prediksi jumlah kendaraan serta perancangan dashboard untuk visualisasi sistem.
Objek detektor akan bertugas untuk melakukan deteksi jenis kendaraan, \textit{tracking} dan perhitungan jumlah kendaraan. LSTM bertugas untuk memprediksi arus lalu lintas berdasarkan data jumlah kendaraan yan telah dibuat oleh objek detektor. Untuk memberikan hasil yang optimal, kedua jaringan akan dilatih dan dievaluasi menggunakan metric nya masing-masing. 
Objek detektor akan dilatih menggunakan dataset sendiri pada \textit{framework} Darknet, dan dievaluasi menggunakan nilai precisions, recall, mAP dan F1-Score. 
Sedangkan LSTM akan dilatih dengan memvariasikan parameter jumlah lapisan LSTM, jumlah neuron tiap lapisan, jumlah data input, serta jumlah data output LSTM dan akan dievaluasi menggunakan data uji untuk mengetahui nilai \textit{Root Mean Squared Error} (RMSE) dan \textit{Mean Average Percentage Error} (MAPE). 

Data dari objek detektor dan LSTM akan ditampilkan jadi satu pada dashboard sehingga informasi dari kedua model dapat dipahami lebih mudah.

\bigskip
\noindent
\textbf{Kata kunci : \textit{You Only Look Once}, \textit{Long-Short Term Memory}, \textit{tracking}, precision, recall , F1-score, \textit{Root Mean Squared Error}, \textit{Mean Average Percentage Error } }
\end{singlespacing}

\chapter*{\emph{INTISARI}}%
%\parindent \z@ \centering 
\normalfont
%{\large\textbf{\abstractname}} \\
\vspace{1.0cm}
\begin{singlespacing}%
%\vskip 20\p@
\addcontentsline{toc}{chapter}{\emph{INTISARI}}

%\emph{}

\bigskip
\noindent

\textbf{\emph{Kata kunci : estimasi titik angkat}, bin picking, \emph{gambar} RGBD, deep learning, semantic segmentation, ROS}
\end{singlespacing}
\end{document}