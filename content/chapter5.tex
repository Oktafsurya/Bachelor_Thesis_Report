% !TEX root = ../thesis.tex
\makeatletter
\def\input@path{{../}}
\makeatother
\documentclass[../thesis.tex]{subfiles}
\begin{document}

\section{Kesimpulan}
Prediksi arus lalu secara \textit{real time} menggunakan kamera CCTV jalan yang berbasis \textit{deep learning} telah dibahas dalam penelitian ini. Secara garis besar penelitian ini dibagi menjadi tiga bagian yaitu
pertama perancangan objek detektor untuk melakukan deteksi, \textit{tracking}, dan perhitungan jumlah kendaraan menggunakan CCTV. Kedua, perancangan model \textit{Long-Short Term Memory} (LSTM) untuk melakukan prediksi arus lalu lintas berdasarkan data yang telah terkumpul. Ketiga, perancangan \textit{dashboard} untuk
visualisasi hasil.

Perancangan objek detektor dilakukan dengan melatihan ulang model YOLOV3 menggunakan dataset yang telah dibuat pada \textit{framework} Darknet. 
Dataset berupa gambar kondisi lalu lintas di Indonesia yang diambil dari layanan \textit{live streaming} CCTV oleh ATCS Semarang. Model YOLOV3 baru hasil pelatihan ulang digunakan untuk
melakukan \textit{tracking} dan perhitungan jumlah kendaraan setiap 5 menit selama 11 hari guna membuat dataset untuk media pelatihan LSTM.

Prediksi arus lalu lintas menggunakan LSTM dilakukan dengan terlebih dahulu melatih model LSTM menggunakan data jumlah kendaraan tiap 5 menit yang telah terkumpul. Untuk mendapatkan model yang optimal, pelatihan model LSTM dilakukan dengan 
memvariasikan beberapa parameter antara lain jumlah input dan output, jumlah lapisan LSTM yang digunakan serta jumlah neuron untuk setiap lapisan. Masing-masing model kemudian dievaluasi untuk melihat besarnya \textit{Root Mean Squared Error} dan 
\textit{Mean Average Percentage Error} nya. Model LSTM optimal yang diperoleh mampu untuk melakukan prediksi arus lalu lintas 1 hari kedepan dengan tren yang menyerupai data pengujian.

Perancangan \textit{dashboard} dilakukan dengan menggunakan aplikasi Dash. Pada perancangan ini dilakukan pembagian proses untuk menghindari penumpukan proses dalam satu perintah. Pengiriman dan penerimaan data antar proses dilakukan menggunakan \textit{library} ZeroMQ. 

\section{Saran}

Untuk meningkatkan performa objek detektor dalam melakukan deteksi,\textit{tracking} dan perhitungan jumlah kendaraan serta kemampuan prediksi model LSTM, beberapa hal yang perlu menjadi
pertimbangan untuk penelitian selanjutnya adalah sebagai berikut:

\begin{enumerate}
\item Dataset yang digunakan untuk melatih ulang model YOLOV3 pada penelitian ini didominasi oleh kelas sepeda motor dan mobil. Hal ini dikarenakan lalu lintas di Indonesia didominasi oleh dua kelas tersebut. Untuk penelitian selanjutnya disarankan agar menambah jumlah data untuk kelas-kelas yang lain agar dataset semakin seimbang.
\item Untuk melakukan perhitungan jumlah kendaran, penelitian ini menetapkan garis acuan terlebih dahulu untuk mengecek apakah garis pada objek/kendaraan bersinggungan atau tidak. Metode seperti ini kurang adaptif terhadap perubahan sudut CCTV yang dikendalikan oleh operator. Untuk penelitian selanjutnya mungkin dapat menggunakan metode lain yang lebih adaptif seperti \textit{road segmentation}.
\item Pada penelitian ini, jumlah data yang digunakan untuk melatih LSTM  hanya sebanyak 11 hari. Sangat disarankan untuk menambah data latih sehingga prediksi yang dihasilkan dapat lebih akurat.
\item Perancangan \textit{dashboard} pada penelitian ini hanya sebatas berjalan di \textit{localhost}. Penelitian selanjutnya dapat mengembangkan \textit{dashboard} untuk siap diimplementasikan di server
\end{enumerate}

\end{document}